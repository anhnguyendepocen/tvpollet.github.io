\documentclass[english,man]{apa6}

\usepackage{amssymb,amsmath}
\usepackage{ifxetex,ifluatex}
\usepackage{fixltx2e} % provides \textsubscript
\ifnum 0\ifxetex 1\fi\ifluatex 1\fi=0 % if pdftex
  \usepackage[T1]{fontenc}
  \usepackage[utf8]{inputenc}
\else % if luatex or xelatex
  \ifxetex
    \usepackage{mathspec}
    \usepackage{xltxtra,xunicode}
  \else
    \usepackage{fontspec}
  \fi
  \defaultfontfeatures{Mapping=tex-text,Scale=MatchLowercase}
  \newcommand{\euro}{€}
\fi
% use upquote if available, for straight quotes in verbatim environments
\IfFileExists{upquote.sty}{\usepackage{upquote}}{}
% use microtype if available
\IfFileExists{microtype.sty}{\usepackage{microtype}}{}

% Table formatting
\usepackage{longtable, booktabs}
\usepackage{lscape}
% \usepackage[counterclockwise]{rotating}   % Landscape page setup for large tables
\usepackage{multirow}		% Table styling
\usepackage{tabularx}		% Control Column width
\usepackage[flushleft]{threeparttable}	% Allows for three part tables with a specified notes section
\usepackage{threeparttablex}            % Lets threeparttable work with longtable

% Create new environments so endfloat can handle them
% \newenvironment{ltable}
%   {\begin{landscape}\begin{center}\begin{threeparttable}}
%   {\end{threeparttable}\end{center}\end{landscape}}

\newenvironment{lltable}
  {\begin{landscape}\begin{center}\begin{ThreePartTable}}
  {\end{ThreePartTable}\end{center}\end{landscape}}

  \usepackage{ifthen} % Only add declarations when endfloat package is loaded
  \ifthenelse{\equal{\string man}{\string man}}{%
   \DeclareDelayedFloatFlavor{ThreePartTable}{table} % Make endfloat play with longtable
   % \DeclareDelayedFloatFlavor{ltable}{table} % Make endfloat play with lscape
   \DeclareDelayedFloatFlavor{lltable}{table} % Make endfloat play with lscape & longtable
  }{}%



% The following enables adjusting longtable caption width to table width
% Solution found at http://golatex.de/longtable-mit-caption-so-breit-wie-die-tabelle-t15767.html
\makeatletter
\newcommand\LastLTentrywidth{1em}
\newlength\longtablewidth
\setlength{\longtablewidth}{1in}
\newcommand\getlongtablewidth{%
 \begingroup
  \ifcsname LT@\roman{LT@tables}\endcsname
  \global\longtablewidth=0pt
  \renewcommand\LT@entry[2]{\global\advance\longtablewidth by ##2\relax\gdef\LastLTentrywidth{##2}}%
  \@nameuse{LT@\roman{LT@tables}}%
  \fi
\endgroup}


\ifxetex
  \usepackage[setpagesize=false, % page size defined by xetex
              unicode=false, % unicode breaks when used with xetex
              xetex]{hyperref}
\else
  \usepackage[unicode=true]{hyperref}
\fi
\hypersetup{breaklinks=true,
            pdfauthor={},
            pdftitle={Are geographically diverse samples used in the journals Evolution and Human Behavior and Evolutionary Psychology?},
            colorlinks=true,
            citecolor=blue,
            urlcolor=blue,
            linkcolor=black,
            pdfborder={0 0 0}}
\urlstyle{same}  % don't use monospace font for urls

\setlength{\parindent}{0pt}
%\setlength{\parskip}{0pt plus 0pt minus 0pt}

\setlength{\emergencystretch}{3em}  % prevent overfull lines

\ifxetex
  \usepackage{polyglossia}
  \setmainlanguage{}
\else
  \usepackage[english]{babel}
\fi

% Manuscript styling
\captionsetup{font=singlespacing,justification=justified}
\usepackage{csquotes}
\usepackage{upgreek}

 % Line numbering
  \usepackage{lineno}
  \linenumbers


\usepackage{tikz} % Variable definition to generate author note

% fix for \tightlist problem in pandoc 1.14
\providecommand{\tightlist}{%
  \setlength{\itemsep}{0pt}\setlength{\parskip}{0pt}}

% Essential manuscript parts
  \title{Are geographically diverse samples used in the journals Evolution and
Human Behavior and Evolutionary Psychology?}

  \shorttitle{Diverse samples}


  \author{Thomas V. Pollet\textsuperscript{1}~\& Tamsin K. Saxton\textsuperscript{1}}

  \def\affdep{{"", ""}}%
  \def\affcity{{"", ""}}%

  \affiliation{
    \vspace{0.5cm}
          \textsuperscript{1} Northumbria University  }

  \authornote{
    \newcounter{author}
    Thomas V. Pollet \& Tamsin K. Saxton., Dept. of Psychology, Northumbria
    University, Newcastle upon Tyne, UK

                      Correspondence concerning this article should be addressed to Thomas V. Pollet, NB 165, Northumberland Building, NE. E-mail: \href{mailto:thomas.pollet@northumbria.ac.uk}{\nolinkurl{thomas.pollet@northumbria.ac.uk}}
                          }


  \abstract{A recurrent criticism is the type of samples used in psychology to draw
inferences. Specifically, researchers have argued that samples are often
restricted to student samples, and that these are also often from
Western Educated Industrialised Rich and Democratic (WEIRD). This
criticism could similarly apply to studies in evolutionary psychology.
However, some researchers have argued that evolutionary psychology
actively recruits much more diverse samples and would thus be less
susceptible to this critique. Here, we empirically examine the samples
used in the 2015 volumes of \emph{Evolution and Human Behavior} (57
articles) and \emph{Evolutionary Psychology} (43 articles). Our database
consists of 171 samples on humans (median sample size= 201). The
majority of samples were either crowd-sourced online or student samples
(62\%), followed by other adult Western samples (19\%). 129 of the
samples can be classified as `Western' (75\%, Europe/North
America/Australia). 26 samples were predominantly from Asia (15\%,
mostly Japan). Only a small fraction of the samples were classified as
cross-cultural (5), African (7), South American (3). The median sample
sizes did not significantly differ between continents but both paid and
unpaid online samples were typically larger than other sample sources.
While it seems that the samples used are more diverse than typical
social psychology samples, it also apparent that the majority of samples
remain WEIRD. We discuss the implications for Evolutionary Psychology as
a discipline.}
  \keywords{Cross-cultural samples; WEIRD; \\

    \indent Word count: X
  }

\usepackage[titles]{tocloft}
\cftpagenumbersoff{figure}
\renewcommand{\cftfigpresnum}{\itshape\figurename\enspace}
\renewcommand{\cftfigaftersnum}{.\space}
\setlength{\cftfigindent}{0pt}
\setlength{\cftafterloftitleskip}{0pt}
\settowidth{\cftfignumwidth}{Figure 10.\qquad}




\usepackage{amsthm}
\newtheorem{theorem}{Theorem}
\newtheorem{lemma}{Lemma}
\theoremstyle{definition}
\newtheorem{definition}{Definition}
\newtheorem{corollary}{Corollary}
\newtheorem{proposition}{Proposition}
\theoremstyle{definition}
\newtheorem{example}{Example}
\theoremstyle{remark}
\newtheorem*{remark}{Remark}
\begin{document}

\maketitle

\setcounter{secnumdepth}{0}



A recurrent criticism of psychology as a science is the lack of
diversity. This lack of diversity refers not solely to who produces
psychological science (e.g., Adair, Coêlho, \& Luna, 2002; Bauserman,
1997; Cole, 2006) or the topics studied (e.g., Berry, 2013) but also
which samples psychology bases its conclusions on. This criticism is
recurrent and has been voiced regularly (Arnett, 2008; Graham, 1992; J
Henrich, Heine, \& Norenzayan, 2010; Henry, 2008; Schultz, 1969; Sears,
1986; e.g., Smart, 1966). While reviewing the subjects used in
psychology Schultz (1969:218) wrote \emph{\enquote{The extremely small
percentage of studies sampling the general adult population was
particularly disturbing; none of the studies published in the Journal of
Experimental Psychology during those years used a sample of the general
population}}. For psychology as a science, to the degree that it tries
to understand the human mind and its universals, it is crucial to
establish that phenomena generalize beyond biased samples, such as
undergraduates (J Henrich et al., 2010; Rozin, 2001). Yet, from its
inception psychology, and perhaps especially social psychology, has
relied heavily on undergraduate samples. The situation does seem to have
improved over time. For example, Schultz (1969) found less than 1.5\% of
the samples of reviewed psychology journals to be from a
\enquote{general adult population}. Sears (1986) reviewed volumes of
social psychology journals and found that 82\% of the samples used
student samples. Gallander Wintre and colleagues reviewed 1,179 articles
spanning 6 journals and all subdivisions of psychology and found 68\% of
the samples to be student samples. They also found that, if anything,
the reliance on student samples had increased from 1975 to 1995. In
1995, two leading social psychology journals (Journal of Experimental
Social Psychology and Journal of Personality and Social Psychology used
undergraduate students as participants in respectively 95.8\% and 70.6\%
of the cases (Gallander Wintre, North, \& Sugar, 2001). Arnett (2008)
tallied that 74\% of the samples in Social Psychological and Personality
Science were student populations. The criticism of a strong reliance on
student samples thus remains and is echoed in various domains of
psychology. The criticism is particularly relevant when dealing with
issues such as socio-political attitudes which may not have been fully
chrystallized in undergraduate populations (Schultz, 1969). (Henry,
2008), for example, criticized the narrow focus on student samples when
examining the psychological processes relating to prejudice. Similarly,
researchers in industrial and organizational psychology have underlined
how the lack of worker samples is problematic when studying
organizational processes (Bergman \& Jean, 2016).

\section{Online participants}\label{online-participants}

Perhaps in part as a response to these criticisms, (social)
psychologists have increasingly turned to online platforms to recruit
participants who were not students (Gosling \& Johnson, 2010). Over the
past decade, there has been a strong increase in the use of crowdsourced
participants via online platforms, such as Amazon M-Turk or Crowdflower
(e.g., Buhrmester, Kwang, \& Gosling, 2011; Paolacci \& Chandler, 2014).
Such expansion has benefited psychology and research shows that results
from classical behavioral experiments (e.g., Stroop Task (Stroop, 1935))
replicate well on these online platforms (Crump, McDonnell, \& Gureckis,
2013). Even for studying political ideologies, it seems that M-Turk is
well-suited (Clifford, Jewell, \& Waggoner, 2015). Yet, perils remain
and thoug Amazon boasts \textgreater{}.5 million participans, the
\emph{actual} pools from which participants are sampled are much smaller
with the population estimated at around 7,300 individuals (Bohannon,
2016). Moreover, while crowd-sourcing platforms such as Amazon's M-Turk,
allow for sampling more diverse samples, for example with regards to age
range, than typical students samples, such samples remain predominantly
WEIRD. Yet, there are some notable exceptions such as (e.g., Raihani,
Mace, \& Lamba, 2013)).

\section{Evolutionary Psychology as the
exception?}\label{evolutionary-psychology-as-the-exception}

The issues relating to sampling are not limited to (social) psychology
and similar concerns have been voiced in consumer research (Peterson,
2001) and business research (Bello, Leung, Radebaugh, Tung, \&
Witteloostuijn, 2009). For example, Peterson (2001) reviewed the
literature in consumer research and found 86\% of the samples to be from
students. Since its inception, evolutionary psychology has stressed the
importance of human universals (e.g., Buss, 1989, 1994; Buss \& Schmitt,
1993; Tooby \& Cosmides, 1990). Empirical examples where researchers
have tested whether universals exist consist of studies of homicide
(Daly \& Wilson, 1988), economic behavior (e.g., Joseph Henrich et al.,
2005) and mate preferences (e.g., Buss, 1989; Buss, Shackelford, \&
LeBlanc, 2000; T. K. Shackelford, Schmitt, \& Buss, 2005).
Cross-cultural universals, for example those listed in (D. E. Brown,
1991, 2000), are also often invoked as evidence for adaptive
psychological mechanisms. It would thus seem that as Apicella and
Barrett (2016, p. 92) have argued that \enquote{\emph{perhaps no field
of psychology is more strongly motivated and better equipped than
evolutionary psychology to respond to the recent call for psychologists
to expand their empirical base beyond WEIRD (Western Educated
Industrialized Rich Democratic) samples}}. Similarly, Kurzban (2013)
argued on the Evolutionary Psychology blog that \emph{\enquote{adding
evolution to psychology makes the science less WEIRD}}. He found that
for the 2012 volume, 65\% of the articles in Evolution and Human
Behavior were WEIRD. In contrast, in the Journal of Personality and
Social Psychology, 96\% of the papers were based on WEIRD samples. This
suggests that evolutionary psychology is indeed less WEIRD than other
subdivisions of psychology. Here we examine the samples used in two
leading evolutionary psychology journals in more depth. We have no
explicit hypotheses but rather describe the samples used in these two
journals.

\section{Methods}\label{methods}

\subsection{Coding}\label{coding}

As part of a larger project eight coders under the supervision of the
first author coded all articles from the journal Evolutionary Psychology
(EP, published by Sage) and Evolution and Human Behavior (E\&HB,
published by Elsevier) for 2015. It should be noted that many key papers
on evolutionary psychology are published in other outlets such as
Journal of Personality and Social Psychology (Buss \& Shackelford, 1997;
e.g., Kenrick, Keefe, Bryan, Barr, \& Brown, 1995) or Psychological
Science (e.g., Buss, Larsen, Westen, \& Semmelroth, 1992). The choice to
limit to these journals was given by the rationale that Kurzban 2013
analyzed and that it was published on the Evolutionary Psychology Blog.
The coders evaluated the type of sample used as well as the continent
from which the data originated based on M49 UNDP codes (United Nations,
2013) (Cross-Cultural, Africa, Asia, Australia (and New-Zealand),
Oceania (other), Europe N., America). When a paper had more than one
sample we coded each sample, when there were more than five samples
After piloting, we settled on the following codes for type of sample:
(Student Sample (Western); Student Sample (Non-Western); Western Child;
Non-Western sample; Other Adult Western sample; Paid Crowdsourced sample
(e.g., M-Turk; Western); Unpaid Crowd-sourced (e.g., a sample recruited
via Twitter)). The division into Western/Non-Western was based on
geographical location, with Europe, Australia and New Zealand, and North
America coded as Western, in line with (Stulp, Simons, Grasman, \&
Pollet, 2017). Where disagreement between coders existed, this was
resolved via discussion.

\subsection{Data analysis}\label{data-analysis}

We used R (3.4.1, R Core Team, 2017) and the R-package \emph{papaja}
(0.1.0.9492, Aust \& Barth, 2017) for all our analyses. We rely on
non-parametric statistics (Siegel \& Castellan, 1988), with post-hoc
comparisons adjusted for multiple testing (Benjamini \& Hochberg, 1995),
given that visual inspection showed that the data were . We use log.
transformations when presenting figures on sample sizes (Keene, 1995).
The data and analysis document are available from the
\href{http://osf.io/pajhy}{Open Science Framework}.

\section{Results}\label{results}

\subsection{Descriptive statistics}\label{descriptive-statistics}

There were 121 articles, of which 100 articles contained codable samples
(EP: 43; E\&HB: 57). There were 171 samples, and the median number of
samples per articles was 1. The mean sample size was 7073.5 but this is
driven by one large sample (N=927134). The median sample size was 201
but sample sizes thus varied substantially (Minimum: 11; First Quartile:
103.5; Third Quartile: 363.5; Maximum: 927134).

Figure 1 shows the distribution of samples. The majority of samples was
from North America (85), followed by Europe (38) and Asia (28). Five
samples are Cross-Cultural. There was one sample from Oceania, Fiji
Island (Mckerracher, Collard, \& Henrich, 2015). The majority of samples
were from Western populations (75\%, Europe/North America/Australia).

46 out of 171 samples were Western student samples, while 17 were
non-Western student samples. Combined this implies that 62\% of the
samples are either online samples or students samples. Only a small
fraction of the samples, consisted of non-Western adults who were not
students (17 out of 171 samples, \textless{} 10\%).

\textbf{insert Figure 1}

\subsection{Are samples from certain geographical locations larger than
others?}\label{are-samples-from-certain-geographical-locations-larger-than-others}

Given that there was but one sample from Oceania we recoded Australia as
Oceania (see \href{http://osf.io/pajhy}{ESM} for additional analyses
where these are not recoded). The sample sizes varied significantly
between continents (Figure 2; Kruskal-Wallis test: \(\chi^2\)(6) =
14.441, \emph{p} = 0.025). Yet, the corrected post-hoc tests showed that
the contrasts of cross-cultural samples versus Africa (\emph{p}=.053),
cross-cultural samples versus Asia (\emph{p}=.07) trended towards
significance, with cross-cultural samples being larger
(\href{http://osf.io/pajhy}{ESM}). Similarly, the contrasts of North
America versus Africa (\emph{p}=.07) and North America versus Asia
(\emph{p}=.07) trended towards significance, with North American samples
being larger. There was no evidence that other contrasts tended to
differ.

\textbf{insert Figure 2}

\subsection{Are some types of samples larger than
others?}\label{are-some-types-of-samples-larger-than-others}

The sample sizes differed significantly according to type (Figure 3;
Kruskal-Wallis test: \(\chi^2\)(7) = 35.015, \emph{p} = 0.0001).
Post-hoc comparisons adjusted for multiple comparisons showed that
unpaid crowdsourced samples tended to be larger than all other types of
samples (all \emph{p}\textless{}.1),

\textbf{insert Figure 3}

\section{Discussion}\label{discussion}

We found 75\% of the samples , which is close to the 2012 number. While,
this number is lower than the numbers typically reported for It should
be noted

TO BE WRITTEN

\section{Acknowledgments}\label{acknowledgments}

The data were collected while the first author was at the University of
Leiden and he wishes to thank his bachelor thesis group for their
support with the project.

\newpage

\section{References}\label{references}

\newpage

Table 1.

Group 1

Group 2

Adjusted p value

1

Paid Crowd

N-W Adult

0.001

2

Unpaid Crowd

N-W Adult

0.001

3

Unpaid Crowd

N-W Student

0.003

4

Unpaid Crowd

W Student

0.003

5

W Student

Paid Crowd

0.003

6

W Adult

N-W Adult

0.018

7

Unpaid Crowd

Paid Crowd

0.018

8

W Student

W Adult

0.019

9

N-W Student

Paid Crowd

0.019

10

Unpaid Crowd

N-W Child

0.062

11

Unpaid Crowd

W Child

0.062

12

Unpaid Crowd

W Adult

0.090

13

N-W Student

W Adult

0.102

14

W Student

N-W Adult

0.163

15

N-W Adult

N-W Child

0.303

16

Paid Crowd

N-W Child

0.303

17

N-W Adult

W Child

0.303

18

N-W Student

N-W Adult

0.360

19

Paid Crowd

W Child

0.380

20

W Adult

N-W Child

0.391

21

W Adult

W Child

0.405

22

Paid Crowd

W Adult

0.554

23

N-W Student

W Child

0.683

24

W Student

W Child

0.683

25

N-W Student

N-W Child

0.711

26

W Student

N-W Child

0.849

27

W Student

N-W Student

0.935

28

W Child

N-W Child

0.953

\newpage

\setlength{\parindent}{-0.5in} \setlength{\leftskip}{0.5in}

\hypertarget{refs}{}
\hypertarget{ref-Adair2002}{}
Adair, J. G., Coêlho, A. E. L., \& Luna, J. R. (2002). How international
is psychology? \emph{International Journal of Psychology}, \emph{37}(3),
160--170.
doi:\href{https://doi.org/10.1080/00207590143000351}{10.1080/00207590143000351}

\hypertarget{ref-Apicella2016a}{}
Apicella, C. L., \& Barrett, H. C. (2016). Cross-cultural evolutionary
psychology. \emph{Current Opinion in Psychology}, \emph{7}, 92--97.
doi:\href{https://doi.org/https://doi.org/10.1016/j.copsyc.2015.08.015}{https://doi.org/10.1016/j.copsyc.2015.08.015}

\hypertarget{ref-Arnett2008}{}
Arnett, J. J. (2008). The neglected 95\%: why American psychology needs
to become less American. \emph{American Psychologist}, \emph{63}(7),
602.

\hypertarget{ref-R-papaja}{}
Aust, F., \& Barth, M. (2017). \emph{papaja: Create APA manuscripts with
R Markdown}. Retrieved from \url{https://github.com/crsh/papaja}

\hypertarget{ref-Bauserman1997}{}
Bauserman, R. (1997). International representation in the psychological
literature. \emph{International Journal of Psychology}, \emph{32}(2),
107--112.
doi:\href{https://doi.org/10.1080/002075997400908}{10.1080/002075997400908}

\hypertarget{ref-Bello2009}{}
Bello, D., Leung, K., Radebaugh, L., Tung, R. L., \& Witteloostuijn, A.
van. (2009). From the Editors: Student samples in international business
research. \emph{Journal of International Business Studies},
\emph{40}(3), 361--364.
doi:\href{https://doi.org/10.1057/jibs.2008.101}{10.1057/jibs.2008.101}

\hypertarget{ref-Benjamini1995}{}
Benjamini, Y., \& Hochberg, Y. (1995). Controlling the False Discovery
Rate: A Practical and Powerful Approach to Multiple Testing.
\emph{Journal of the Royal Statistical Society. Series B
(Methodological)}, \emph{57}(1), 289--300. Retrieved from
\url{http://www.jstor.org/stable/2346101}

\hypertarget{ref-Bergman2016}{}
Bergman, M. E., \& Jean, V. A. (2016). Where have all the ``workers''
gone? A critical analysis of the unrepresentativeness of our samples
relative to the labor market in the industrial--organizational
psychology literature. \emph{Industrial and Organizational Psychology},
\emph{9}(1), 84--113.
doi:\href{https://doi.org/10.1017/iop.2015.70}{10.1017/iop.2015.70}

\hypertarget{ref-Berry2013}{}
Berry, J. W. (2013). Achieving a global psychology. \emph{Canadian
Psychology/Psychologie Canadienne}, \emph{54}(1), 55--61.
doi:\href{https://doi.org/10.1037/a0031246}{10.1037/a0031246}

\hypertarget{ref-Bohannon2016}{}
Bohannon, J. (2016, June). Psychologists grow increasingly dependent on
online research subjects.
doi:\href{https://doi.org/10.1126/science.aag0592}{10.1126/science.aag0592}

\hypertarget{ref-Brown1991}{}
Brown, D. E. (1991). \emph{Human universals}. New York, NY: McGraw-Hill.

\hypertarget{ref-Brown2000}{}
Brown, D. E. (2000). Human universals and their implications. In N.
Roughley (Ed.), \emph{Being humans: Anthropological universality and
particularity in transdisciplinary perspectives} (pp. 156--174). Berlin:
Walter de Gruyter.

\hypertarget{ref-Buhrmester2011}{}
Buhrmester, M., Kwang, T., \& Gosling, S. D. (2011). Amazon's Mechanical
Turk: A New Source of Inexpensive, Yet High-Quality, Data?
\emph{Perspectives on Psychological Science}, \emph{6}(1), 3--5.
doi:\href{https://doi.org/10.1177/1745691610393980}{10.1177/1745691610393980}

\hypertarget{ref-Buss1989}{}
Buss, D. M. (1989). Sex differences in human mate preferences:
Evolutionary hypotheses tested in 37 cultures. \emph{Behavioral and
Brain Sciences}, \emph{12}(1), 1--49.
doi:\href{https://doi.org/10.1017/S0140525X00023992}{10.1017/S0140525X00023992}

\hypertarget{ref-Buss1994}{}
Buss, D. M. (1994). \emph{The evolution of desire: Strategies of human
mating.} New York, NY: Basic books.

\hypertarget{ref-Buss1993}{}
Buss, D. M., \& Schmitt, D. P. (1993). Sexual strategies theory: An
evolutionary perspective on human mating. \emph{Psychological Review},
\emph{100}(2), 204--232.
doi:\href{https://doi.org/10.1037/0033-295X.100.2.204}{10.1037/0033-295X.100.2.204}

\hypertarget{ref-Buss1997}{}
Buss, D. M., \& Shackelford, T. K. (1997). From vigilance to violence:
Mate retention tactics in married couples. \emph{Journal of Personality
and Social Psychology}, \emph{72}(2), 346--361. Article.
doi:\href{https://doi.org/10.1037/0022-3514.72.2.346}{10.1037/0022-3514.72.2.346}

\hypertarget{ref-Buss1992}{}
Buss, D. M., Larsen, R. J., Westen, D., \& Semmelroth, J. (1992). Sex
Differences in Jealousy: Evolution, Physiology, and Psychology.
\emph{Psychological Science}, \emph{3}(4), 251--255.
doi:\href{https://doi.org/10.1111/j.1467-9280.1992.tb00038.x}{10.1111/j.1467-9280.1992.tb00038.x}

\hypertarget{ref-Buss2000}{}
Buss, D. M., Shackelford, T. K., \& LeBlanc, G. J. (2000). Number of
children desired and preferred spousal age difference: context-specific
mate preference patterns across 37 cultures. \emph{Evolution and Human
Behavior}, \emph{21}(5), 323--331.
doi:\href{https://doi.org/10.1016/S1090-5138(00)00048-9}{10.1016/S1090-5138(00)00048-9}

\hypertarget{ref-Clifford2015}{}
Clifford, S., Jewell, R. M., \& Waggoner, P. D. (2015). Are samples
drawn from Mechanical Turk valid for research on political ideology?
\emph{Research \& Politics}, \emph{2}(4), 205316801562207.
doi:\href{https://doi.org/10.1177/2053168015622072}{10.1177/2053168015622072}

\hypertarget{ref-Cole2006}{}
Cole, M. (2006). Internationalism in psychology: We need it now more
than ever. \emph{American Psychologist}, \emph{61}(8), 904--917.
doi:\href{https://doi.org/10.1037/0003-066X.61.8.904}{10.1037/0003-066X.61.8.904}

\hypertarget{ref-Crump2013}{}
Crump, M. J. C., McDonnell, J. V., \& Gureckis, T. M. (2013). Evaluating
Amazon's Mechanical Turk as a Tool for Experimental Behavioral Research.
\emph{PLOS ONE}, \emph{8}(3), e57410. Retrieved from
\url{https://doi.org/10.1371/journal.pone.0057410}

\hypertarget{ref-Daly1988}{}
Daly, M., \& Wilson, M. (1988). \emph{Homicide}. New Brunswick, NJ:
Transaction Books.

\hypertarget{ref-Gallander2001}{}
Gallander Wintre, M., North, C., \& Sugar, L. A. (2001). Psychologists'
response to criticisms about research based on undergraduate
participants: A developmental perspective. \emph{Canadian
Psychology/Psychologie Canadienne}, \emph{42}(3), 216--225.
doi:\href{https://doi.org/10.1037/h0086893}{10.1037/h0086893}

\hypertarget{ref-Gosling2010a}{}
Gosling, S. D., \& Johnson, J. a. (2010). \emph{Advanced Methods for
Conducting Online Behavioral Research} (p. 286).
doi:\href{https://doi.org/10.1037/12076-000}{10.1037/12076-000}

\hypertarget{ref-Graham1992}{}
Graham, S. (1992). `` Most of the subjects were White and middle
class'': Trends in published research on African Americans in selected
APA journals, 1970--1989. \emph{American Psychologist}, \emph{47}(5),
629--639.

\hypertarget{ref-Henrich2005}{}
Henrich, J., Boyd, R., Bowles, S., Camerer, C., Fehr, E., Gintis, H.,
\ldots{} Tracer, D. (2005). ``Economic man'' in cross-cultural
perspective: Behavioral experiments in 15 small-scale societies.
\emph{Behavioral and Brain Sciences}, \emph{28}(06), 795--815.
doi:\href{https://doi.org/10.1017/S0140525X05000142}{10.1017/S0140525X05000142}

\hypertarget{ref-Henrich2010}{}
Henrich, J., Heine, S. J., \& Norenzayan, A. (2010). The weirdest people
in the world. \emph{Behavioral and Brain Sciences}, \emph{33}(2-3),
61--83.
doi:\href{https://doi.org/10.1017/S0140525X0999152X}{10.1017/S0140525X0999152X}

\hypertarget{ref-Henry2008}{}
Henry, P. J. (2008). College sophomores in the laboratory redux:
Influences of a narrow data base on social psychology's view of the
nature of prejudice. \emph{Psychological Inquiry}, \emph{19}(2), 49--71.
doi:\href{https://doi.org/10.1080/10478400802049936}{10.1080/10478400802049936}

\hypertarget{ref-Keene1995}{}
Keene, O. N. (1995). The log transformation is special. \emph{Statistics
in Medicine}, \emph{14}(8), 811--819.
doi:\href{https://doi.org/10.1002/sim.4780140810}{10.1002/sim.4780140810}

\hypertarget{ref-Kenrick1995}{}
Kenrick, D. T., Keefe, R. C., Bryan, A., Barr, A., \& Brown, S. (1995).
Age preferences and mate choice among homosexuals and heterosexuals: A
case for modular psychological mechanisms. \emph{Journal of Personality
and Social Psychology}, \emph{69}(6), 1166.
doi:\href{https://doi.org/10.1037/0022-3514.69.6.1166}{10.1037/0022-3514.69.6.1166}

\hypertarget{ref-Kurzban2013a}{}
Kurzban, R. (2013). Is Evolutionary Psychology WEIRD or NORMAL?
Retrieved from
\url{http://epjournal.net/blog/2013/09/is-evolutionary-psychology-weird-or-normal/}

\hypertarget{ref-Mckerracher2015}{}
Mckerracher, L., Collard, M., \& Henrich, J. (2015). The expression and
adaptive significance of pregnancy-related nausea, vomiting, and
aversions on Yasawa Island, Fiji. \emph{Evolution and Human Behavior},
\emph{36}(2), 95--102.
doi:\href{https://doi.org/https://doi.org/10.1016/j.evolhumbehav.2014.09.005}{https://doi.org/10.1016/j.evolhumbehav.2014.09.005}

\hypertarget{ref-Paolacci2014}{}
Paolacci, G., \& Chandler, J. (2014). Inside the turk understanding
mechanical turk as a participant pool. \emph{Current Directions in
Psychological Science}, \emph{23}(3), 184--188.
doi:\href{https://doi.org/10.1177/0963721414531598}{10.1177/0963721414531598}

\hypertarget{ref-Peterson2001}{}
Peterson, R. A. (2001). On the Use of College Students in Social Science
Research: Insights from a Second-Order Meta-analysis. \emph{Journal of
Consumer Research}, \emph{28}(3), 450--461.
doi:\href{https://doi.org/10.1086/323732}{10.1086/323732}

\hypertarget{ref-R-base}{}
R Core Team. (2017). \emph{R: A language and environment for statistical
computing}. Vienna, Austria: R Foundation for Statistical Computing.
Retrieved from \url{https://www.R-project.org/}

\hypertarget{ref-Raihani2013}{}
Raihani, N. J., Mace, R., \& Lamba, S. (2013). The Effect of \$1, \$5
and \$10 Stakes in an Online Dictator Game. \emph{PLoS ONE},
\emph{8}(8), e0073131.
doi:\href{https://doi.org/10.1371/journal.pone.0073131}{10.1371/journal.pone.0073131}

\hypertarget{ref-Rozin2001}{}
Rozin, P. (2001). Social psychology and science: Some lessons from
Solomon Asch. \emph{Personality and Social Psychology Review},
\emph{5}(1), 2--14.

\hypertarget{ref-Schultz1969}{}
Schultz, D. P. (1969). The human subject in psychological research.
\emph{Psychological Bulletin}, \emph{72}(3), 214.

\hypertarget{ref-Sears1986}{}
Sears, D. O. (1986). College sophomores in the laboratory: Influences of
a narrow data base on social psychology's view of human nature.
\emph{Journal of Personality and Social Psychology}, \emph{51}(3),
515--530.

\hypertarget{ref-Shackelford2005}{}
Shackelford, T. K., Schmitt, D. P., \& Buss, D. M. (2005). Universal
dimensions of human mate preferences. \emph{Personality and Individual
Differences}, \emph{39}(2), 447--458.
doi:\href{https://doi.org/10.1016/j.paid.2005.01.023}{10.1016/j.paid.2005.01.023}

\hypertarget{ref-Siegel1988}{}
Siegel, S., \& Castellan, N. J. (1988). \emph{Nonparametric statistics
for the behavioral sciences.} (2nd ed.). New York, NY: McGraw-hill.

\hypertarget{ref-Smart1966}{}
Smart, R. G. (1966). Subject selection bias in psychological research.
\emph{Canadian Psychologist/Psychologie Canadienne}, \emph{7a}(2),
115--121. doi:\href{https://doi.org/10.1037/h0083096}{10.1037/h0083096}

\hypertarget{ref-Stroop1935}{}
Stroop, J. R. (1935). Studies of interference in serial verbal
reactions. \emph{Journal of Experimental Psychology}, \emph{18}(6),
643--662. doi:\href{https://doi.org/10.1037/h0054651}{10.1037/h0054651}

\hypertarget{ref-Stulp2017}{}
Stulp, G., Simons, M., Grasman, S., \& Pollet, T. (2017). Assortative
mating for human height: A meta-analysis. \emph{American Journal of
Human Biology}, \emph{29}(1).
doi:\href{https://doi.org/10.1002/ajhb.22917}{10.1002/ajhb.22917}

\hypertarget{ref-Tooby1990}{}
Tooby, J., \& Cosmides, L. (1990). The past explains the present:
Emotional adaptations and the structure of ancestral environments.
\emph{Ethology and Sociobiology}, \emph{11}(4), 375--424.
doi:\href{https://doi.org/10.1016/0162-3095(90)90017-Z}{10.1016/0162-3095(90)90017-Z}

\hypertarget{ref-UnitedNations2013a}{}
United Nations. (2013). Composition of macro geographical (continental)
regions, geographical sub-regions, and selected economic and other
groupings. Retrieved from
\url{https://unstats.un.org/unsd/methods/m49/m49regin.htm}



\clearpage
\renewcommand{\listfigurename}{Figure captions}
\listoffigures



\end{document}
